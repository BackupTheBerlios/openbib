%\documentclass[11pt]{scrartcl}
\documentclass[11pt, twoside, a4paper, BCOR8mm, DIV12, bibtotoc,idxtotoc]{scrreprt}
\usepackage{german}
\usepackage{typearea}
\usepackage{longtable}
\usepackage{hyperref}
\usepackage{graphicx}

% Zusaetzliche Picture-Umgebungen (z.B. shadowenv)
\usepackage{picins}

% Header anpassbar
\usepackage{fancyhdr}

% Headings umdefinieren
\pagestyle{fancy}
\fancyhf{}
\fancyhead[RO]{\nouppercase{\rightmark}}
\fancyhead[LE]{\nouppercase{\leftmark}}
\fancyfoot[RO, LE]{\thepage}

%\addtolength{\headwidth}{\marginparsep}
%\addtolength{\headwidth}{\marginparwidth}
\addtolength{\headwidth}{1cm}

\parindent0.0mm
\parskip0.3cm    
\typearea{13}

\begin{document}

\begin{titlepage}

\begin{center}
\rule[-.1in]{16cm}{1mm}\\[3mm]
{\fontfamily{cmss}\fontseries{bx}\fontshape{n}\fontsize{20}{20pt}\selectfont
  www.openbib.org $\bullet$ OpenBib Rechercheportal}\\[-2mm]
\rule[-.1in]{16cm}{1mm}

\vspace{5cm}

  \textbf{\fontfamily{cmss}\fontseries{bx}\fontshape{n}\fontsize{30}{30pt}\selectfont Open Digital Information/Image Archive (OpenDIA)\\[3mm] Workflow-Konzept}

  \vspace{2cm}

  Oliver Flimm \texttt{<flimm@openbib.org>}\\
  Stand: \today

  \vspace{7cm}

\rule[-.1in]{16cm}{1mm}\\[3mm]
{\fontfamily{cmss}\fontseries{bx}\fontshape{n}\fontsize{20}{20pt}\selectfont
  www.openbib.org $\bullet$ OpenBib Rechercheportal}\\[-2mm]
\rule[-.1in]{16cm}{1mm}

\end{center}

\end{titlepage}

%\thispagestyle{empty}

%\begin{verbatim}


%Copyright (c) 2005 Oliver Flimm <flimm@openbib.org>

%Es wird die Erlaubnis gegeben dieses Dokument zu kopieren, verteilen 
%und/oder zu veraendern unter den Bedingungen der GNU Free
%Documentation License, Version 1.1 oder einer spaeteren, von der Free 
%Software Foundation veroeffentlichten Version; mit den
%Unveraenderlichen Abschnitten DEREN TITEL AUFGEZAEHLT sind, mit den 
%Vorderseitentexten die AUFGEZAEHLT sind, und mit den Rueckseitentexten
%die AUFGEZAEHLT sind. Eine Kopie dieser Lizenz ist in dem Abschnitt 
%enthalten, der mit "GNU Free Documentation License"
%\end{verbatim}

\tableofcontents

\chapter{Aufgabe}
Die zu bew"altigende Aufgabe besteht in der Erstellung, Beschreibung,
Darstellung und Ver\-brei\-tung von Digitalisat-Serien. Bei den
Digitalisaten handelt es sich prim"ar um Folgen von digital
ein\-ge\-scann\-ten Bildern, die eine logische Einheit bilden - z.B. ein
Buch, einen Einband, ein Akzessions\-jour\-nal usw. In sich k"onnen diese
Digitalisate eine hierarchische Struktur mit verschiedenen
Ord\-nungs\-einheiten aufweisen, wie z.B. Kapitel, Sektionen,
Untersektionen usw.

Mit diesem Papier soll ein Konzept vorgestellt werden, wie ein
Einfachstansatz f"ur einen Workflow realisiert werden kann, mit dem
diese Aufgaben zu bew"altigen sind.

Grunds"atzlich sind in dem Workflow folgende Schritte abzudecken:
\begin{description}
\item[Scan-Vorgang] In diesem Schritt werden die
  Bild-Dateien in einer gr"o"stm"oglichen -- aber dennoch sinnvollen
  -- Aufl"osung erzeugt. Diese Daten m"ussen gespeichert werden.
\item[Anreicherung mit Meta-Daten] Zusammengeh"orige Bild-Dateien
  bilden ein logisches Digitalisat und m"ussen mit Meta-Daten
  angereichert werden.  Diese k"onnen das gesamte Digitalisat,
  einzelne Ordnungs\-einheiten (z.B. Kapitel usw. mit festgelegter
  Abfolge) oder einzelne Bilder beschreiben und m"ussen u.a. auch die
  Gesamt-Struktur der Bilder bzgl.  Ordnungseinheiten und Abfolgen
  beschreiben (entsprechend Inhaltsverzeichnissen/Ebe\-nen).
\end{description}

Damit ist der Hauptteil der zu leistenden Arbeit abgedeckt. Die
Anreicherung mit Meta-Daten bindet das meiste Personal und stellt den
aufw"andigsten Teil dar. Nachdem alle Meta-Daten erfasst sind, folgen
diese Schritte:
\begin{description}
\item[Pr"asentation] Die Bilder m"ussen mit ihren Meta-Daten und ihrer
  Ordnungsstruktur im Web pr"asentiert werden. Das stellt ein geringes
  Problem dar, da sich prinzipiell programm\-tech\-nisch alle denkbaren
  Browsing/Darstellungsarten anhand der Metadaten und der Bilder
  erzeugen lassen. Denkbar sind sowohl teils statische Webseiten, die
  im Voraus erzeugt werden m"ussen, oder dynamisch just-in-time
  generierte Webseiten, die durch ein geeignetes Programm angezeigt
  werden.
\item[Integration] Durch geeignete Mechanismen muss eine Integration
  in andere Dienste -- unabh"angig von einer angestrebten
  eigenst"andigen Nutzbarkeit -- m"oglich sein. Hierzu geh"ort die
  lokale Integration in Nachweisinstrumente wie den KUG, aber auch die
  Integration in externe Systeme, z.B. via OAI oder WebServices.
\end{description}

\chapter{Grundkonzept}

\section{Anforderungen}

\subsection{Flexibilit"at}
Die zu bew"altigende Aufgabe ist bezogen auf ihre wesentlichen
Schritte sehr klar strukturiert. Dennoch kann f"ur jeden dieser
Schritte der Teufel im Detail stecken. Erfahrungsgem"a"s werden erst
nach und nach konrete Anforderungen (z.B. speziell das zu verwendende
Meta-Datenformat) formuliert, so da"s davon auszugehen ist, dass die
erste Version in der Realisierung nicht die letzte bleibt. Daher ist
eine maximale Flexibilit"at der anvisierten L"osung unabdingbar.


\subsection{Einfachheit in der Bedienung und Administration}
Die L"osung mu"s letztlich durch das vorhandene Personal
administriert, angepasst, erweitert und bedient werden. Aus diesem
Grund sollte sie 
\begin{itemize}
\item auf bereits Bekanntem aufsetzen und
\item keine neuen, nicht beherrschten Technologien einsetzen. 
\end{itemize}

\subsection{Kosten}
Ebenso spielen die Kosten eine wesentliche Rolle. Kommerzielle
L"osungen sind in der Regel nicht sehr preiswert. Durch die Verwendung
von bereits vorhandener bekannter Software und bekannten Technologien
sowie von OpenSource-Produkten k"onnen die Kosten minimiert werden.


\section{Konzept}
Unter Ber"ucksichtigung der Anforderungen kann ein Workflow wie folgt
realisiert werden.

\subsection{Scan-Vorgang}
Die Dateien werden vom zust"andigen Personal gescannt und in einem
Verzeichnisbaum auf einem Netz-Laufwerk \emph{struk\-tu\-riert und
  darin in der jeweils logischen Abfolge} entsprechend vorher
getroffener Vereinbarungen gespeichert. Strukturiert bedeutet eine
Verzeich\-nis\-struk\-tur auf dem Laufwerk in der Form
\begin{verbatim}
.../collection/digitalisat/kapitel/unterkapitel/etc.
\end{verbatim}
Die generelle Organisation erfolgt in Digitalisat-Serien
(Collections), in denen Digitalisate abgelegt werden. 

Durch die Verwendung von Verzeichnissen unterhalb \texttt{digitalisat}
lassen sich hierarchische Struk\-tu\-ren des Dokuments allein mit den
Mittel eines Dateisystems abbilden. So k"onnten schon beim Scan die
Verzeichnisse, Unterverzeich\-nis\-se etc. auf dem Laufwerk erzeugt
werden, die dann einem Kapitel, Unterkapitel etc.  entsprechen. Wenn
keine (Unter)kapitel etc.  existieren, dann vereinfacht sich die
Organisation der Bild-Daten entsprechend.


Beispiel ohne Kapitel f"ur die Digitalisat-Serie 'einbaende':
\begin{verbatim}
.../einbaende/1/0001.tif
.../einbaende/1/0002.tif
.../einbaende/1/0003.tif
.../einbaende/1/0004.tif
.../einbaende/1/0005.tif
\end{verbatim}

Beispiel mit Kapitel f"ur die Digitalisat-Serie 'einbaende':
\begin{verbatim}
.../einbaende/1/0001.tif
.../einbaende/1/0002.tif
.../einbaende/1/kap_1/0003.tif
.../einbaende/1/kap_1/0004.tif
\end{verbatim}

Bei Digitalisaten aus katalogisierten B"uchern entspr"ache
\texttt{digitalisat} sinnvollerweise dem nu\-mer\-ischen
Katalogschl"ussel. In den Beispielen w"are der entsprechende
Katalogschl"ussel in der zugeh"origen Katalog-Datenbank, in den
Beispielen z.B. 1. "Uber eine solche Vereinbarung w"are eine
Daten"uber\-nahme aus den Katalogdaten problemlos m"oglich.

Die zu verwendenden Werkzeuge dieses Schrittes sind bereits vorhanden
und in der Bedienung bekannt. Es ist der Scanner samt Software und ein
Datei-Browser (Windows Explorer oder KDE Konqueror).

Als Ergebnis dieses Schrittes liegen die Bild-Daten in der korrekten
Reihenfolge und Strukturierung auf dem Netz-Laufwerk vor.


\subsection{Anreicherung mit Meta-Daten}
F"ur das Anlegen von Meta-Daten (und zus"atzliche Arbeiten f"ur die
Web-Pr"asentation, die auch hier angesiedelt/angestossen werden
k"onn(t)en) sind verschiedene Modelle denkbar. Ein solches Modell soll
nun vorgestellt werden.


\subsubsection{Meta-Daten-Mapping "uber das Datei-System}
Der Grundgedanke ist: \emph{Meta-Informationen werden in Dateien mit
  geeigneten Namen an ge\-eig\-ne\-ter Stelle abgespeichert, so dass eine
  Zuordnung zu einer Digitalisat-Serie, einem Gesamt\-di\-gi\-ta\-li\-sat,
  (Unter)Kapitel oder Bild m"oglich ist.}

Das bedeutet in der konkreten Realisierung:
\begin{itemize}
\item Alle Meta-Daten werden in Dateien abgelegt, die eine spezifische
  Endung (z.B. \texttt{.dsc}) besitzen.
\item Pro Kategorie eines Meta-Datums zu einem Bezugs-Objekt
  (Digitalisat-Serie, einzelnes Digitalisat, einzelnes Bild einzelnes
  Kapitel etc.) wird eine Datei verwendet.
\item Jede Meta-Datenkategorie wird im Meta-Daten-Dateinamen
  kodiert. 
\item Die Dateien von Meta-Daten zu einer hierarchischen Ebene
  (Digitalisat-Serie, einzelnes Digitalisat, Kapitel etc.) beginnen
  mit \texttt{meta},  an welches getrennt durch ein
  definiertes Zeichen (z.B.  \texttt{\_}) der Name der
  Meta-Datenkategorie angeh"angt wird.
\item Die Dateien von Meta-Daten zu einer (Bild-)Datei beginnen mit
  dem zugrunde liegenden (Bild-)Dateinamen, an den getrennt durch ein
  definiertes Zeichen (z.B.  \texttt{\_}) der Name der
  Meta-Datenkategorie angeh"angt wird.
\item Multiple Inhalte einer Meta-Datenkategorie werden in der
  entsprechenden Datei voneinander getrennt (z.B. durch '\texttt{. - }')
\end{itemize}

\textbf{Beispiel:} Willk"urliche Meta-Information 'Beschreibung' via
Meta-Datenkategorie DC.Subject

Beschreibung des Gesamtdokuments/Digitalisates:
\begin{verbatim}
.../einbaende/1/meta_DC.Subject.dsc 
\end{verbatim}

Beschreibung eines Bildes 0001.tif:
\begin{verbatim}
.../einbaende/1/0001.tif_DC.Subject.dsc
\end{verbatim}

Beschreibung des Kapitels 1:
\begin{verbatim}
.../einbaende/1/kap_1/meta_DC.Subject.dsc
\end{verbatim}

Beschreibung des Bildes 0003.tif in Kapitel 1:
\begin{verbatim}
.../einbaende/1/kap_1/0003.tif_DC.Subject.dsc
\end{verbatim}

Ein gro"ser Vorteil dieses Ansatzes ist
\begin{itemize}
\item die M"oglichkeit der sofortigen Verwendung beliebiger
  Meta-Datenkategorien, ohne das zugrunde liegende Datenmodell jeweils
  dem neuen Meta-Datenmodell anpassen zu m"ussen,
\item die M"oglichkeit auch nachtr"aglich -- nach der Vergabe von
  Meta-Daten (s.u.) -- die Struktur des Dokuments durch einfache Operationen
  des Datei-Browsers auf dem Netzwerklaufwerk zu "andern.
\end{itemize}

\subsubsection{Web-basierter Editor f"ur die Meta-Datenverarbeitung}
Der Web-basierte Editor greift grund"atzlich auf den
Festplattenbereich zu, auf dem im voran\-ge\-gan\-ge\-nen Schritt die
Scans abgelegt wurde. Damit kann die Web-Schnittstelle von sich aus
schon auf die so strukturierten Bild-Dateien zugreifen.

Der Benutzer bekommt in einer Auswahl alle 
Digitalisat-Serienverzeichnisse 
\begin{verbatim}
.../collection/...
\end{verbatim}
auf dem Netz-Laufwerk angeboten, von der
er die Digitalisat-Serie (collection) ausw"ahlt, die er
bearbeiten m"ochte.

Daraufhin bekommt er die vorhandenen einzelnen Digitalisate
\begin{verbatim}
.../.../digitalisat/...
\end{verbatim}
an\-ge\-zeigt, von denen er sich eines
ausw"ahlen kann. Unabh"angig davon kann der Benutzer an dieser Stelle
Meta-Daten f"ur die Digitalisat-Serie vergeben.

Daraufhin bekommt er eine Darstellung der Dokumentbestandteile/struktur
angezeigt. 

An dieser Stelle kann der Benutzer die entsprechenden Meta-Daten f"ur
das Digitalisat als Ganzes, einzelne Bestandteile oder
Ordnungseinheiten eingeben.

Zus"atzlich k"onnten an dieser Stelle verschiedene Schritte
angesiedelt sein, die f"ur die Web-Ver"offentlichung relevant sind und
von hier aus anstossen werden k"onnten. Dazu k"onnte die automatische
Erzeugung von Pictogrammen/Thumbnails der gescannten Bilder sowie eine
oder mehrere spezielle Versionen f"ur die Web-Darstellung geh"oren.

Hier wird auch die M"oglichkeit der Daten"uber\-nahme aus einem
bereits bestehenden Katalog angeboten werden.

Die Abspeicherung der hier eingegebenen Metadaten erfolgt in dem oben
dargelegten Text-Dateischema.  Eine direkte Speicherung in einer
XML-Strukur w"are ebenso m"oglich. Gleiches gilt f"ur den Aufruf eines
schon bearbeiteten Digitalisats. Die vorher aufgenommenen
Meta-Informationen k"onnen aus dem Text-Dateischema extrahiert werden,
oder aus einer vorher ab\-ge\-spei\-cher\-ten XML-Datei.

Ebenso k"onnen hier automatisch URN's generiert, an die DDB gemeldet
werden usw. Hier ist fast alles automatisierbare ansiedelbar.

Dieser Ansatz bietet folgende Vorteile:
\begin{itemize}
\item Einfache Bedienung
\item Einfache Eingabem"oglichkeit auch vieler (gleichartiger)
Meta-Informationen 'auf einen Rutsch'
\item Kopplung mit dem Katalogsystem problemlos m"oglich zur
automatischen Meta-Daten\-"uber\-nahme bzw. -er\-zeugung.
\item Maximale Flexibilit"at
\end{itemize}


\subsection{Pr"asentation}
Ausgehend von den Metadaten und den Bildern wird ein Programm
erstellt, um die Digitalisate zu pr"asentieren. Hier ist die
Integration einer MySQL-Datenbank sinnvoll, in die nach Fertigstellung
eines Digitalisats oder einer Digitalisie\-rungs-Serie die
entsprechenden Informationen -- Meta-Daten und Scan-Daten -- abgelegt
werden k"onnen, um diese dann bei der Pr"asentation (z.B. beim
Browsing) schnell verwenden zu k"onnen. Generell ist dieser Teil --
wenn die Meta-Daten einmal existieren -- sehr einfach zu realisieren.

\subsection{Integration}
Durch die Verwendung der MySQL-Datenbank sind aus den dort
angesiedelten Informationen alle denkbaren Ausgabeformate
generierbar. Diese k"onnen XML-basiert sein und z.B. via OAI angeboten
werden. Ebenso ist durch die Offenheit der Anwendung eine Integration
in andere Systeme problemlos m"oglich.

\subsection{Fazit}
Der dargestellte Einfachstansatz eines Workflow-Konzeptes auf
Grundlage des verwendeten Map\-pings in das Dateisystem
\begin{verbatim}
Scannen -> Meta-Daten Editor -> Praesentation/Integration
\end{verbatim}
ist beliebig modifizierbar, soll aber im Wesentlichen zeigen, dass
alle Phasen des Workflows auch mit sehr einfachen Bordmitteln (Scanner
samt Program, Datei-Browser und Web-Browser) und etwas Programmierung
bew"altigt werden k"onnen.

Der Ansatz ist so offen, da"s beliebige Meta-Datenformate (auch
parallel) f"ur beliebige Digitalisie\-rungs-Serien realisiert werden
k"onnen. Jedes Meta-Datenformat mu"s lediglich in einer
Kon\-fi\-gu\-ra\-tions\-datei mit seiner Struktur abgelegt werden. Eine
"Anderung der Programme, des Datenmodells usw. ist nicht notwendig.

Dieser Ansatz realisiert damit genau die an die L"osung gestellten
Anforderungen:
\begin{itemize}
\item Simpel und daher auch einfach zu administrieren, erweitern und
  bedienen.
\item Maximale Flexibilit"at (Anpass- u. Erweiterbarkeit)
\item Minimale Kosten
\end{itemize}

\end{document}
