%\documentclass[11pt]{scrartcl}
\documentclass[11pt, twoside, a4paper, BCOR8mm, DIV12, bibtotoc,idxtotoc]{scrbook}
\usepackage{german}
\usepackage{typearea}
\usepackage{longtable}
\usepackage{hyperref}
\usepackage{graphicx}
\usepackage[latin1]{inputenc}

% Zusaetzliche Picture-Umgebungen (z.B. shadowenv)
\usepackage{picins}

% Header anpassbar
\usepackage{fancyhdr}

% Headings umdefinieren
\pagestyle{fancy}
\fancyhf{}
\fancyhead[RO]{\nouppercase{\rightmark}}
\fancyhead[LE]{\nouppercase{\leftmark}}
\fancyfoot[RO, LE]{\thepage}

%\addtolength{\headwidth}{\marginparsep}
%\addtolength{\headwidth}{\marginparwidth}
\addtolength{\headwidth}{1cm}

\parindent0.0mm
\parskip0.3cm    
\typearea{13}

\begin{document}

\frontmatter

\begin{titlepage}

\begin{center}
\rule[-.1in]{16cm}{1mm}\\[3mm]
{\fontfamily{cmss}\fontseries{bx}\fontshape{n}\fontsize{20}{20pt}\selectfont
  www.openbib.org $\bullet$ Open Library WebServices}\\[-2mm]
\rule[-.1in]{16cm}{1mm}

\vspace{5cm}

  \textbf{\fontfamily{cmss}\fontseries{bx}\fontshape{n}\fontsize{30}{30pt}\selectfont Open Library WebServices\\[3mm] Technische Dokumentation}

  \vspace{2cm}

  Oliver Flimm \texttt{<flimm@openbib.de>}\\
  Stand: \today

  \vspace{8cm}

\rule[-.1in]{16cm}{1mm}\\[3mm]
{\fontfamily{cmss}\fontseries{bx}\fontshape{n}\fontsize{20}{20pt}\selectfont
  www.openbib.org $\bullet$ Open Library WebServices}\\[-2mm]
\rule[-.1in]{16cm}{1mm}

\end{center}

\end{titlepage}

%\thispagestyle{empty}

%\begin{verbatim}


%Copyright (c) 2005 Oliver Flimm <flimm@openbib.org>

%Es wird die Erlaubnis gegeben dieses Dokument zu kopieren, verteilen 
%und/oder zu veraendern unter den Bedingungen der GNU Free
%Documentation License, Version 1.1 oder einer spaeteren, von der Free 
%Software Foundation veroeffentlichten Version; mit den
%Unveraenderlichen Abschnitten DEREN TITEL AUFGEZAEHLT sind, mit den 
%Vorderseitentexten die AUFGEZAEHLT sind, und mit den Rueckseitentexten
%die AUFGEZAEHLT sind. Eine Kopie dieser Lizenz ist in dem Abschnitt 
%enthalten, der mit "GNU Free Documentation License"
%\end{verbatim}

\tableofcontents

\mainmatter

\chapter{Hintergrund und Zielsetzung}

Ein gro�es Problem in der derzeitigen Bibliothekslandschaft ist die
Zersplitterung der angebotenen Dienste. Typischerweise bildet ein
kommerzielles Bibliothekssystem mit Diensten wie Katalogi\-sier\-ung,
Erwerbung, Ausleihe und Recherche �ber einen WebOPAC die Basis, an die
andere, meist externe Dienste wie zus�tzliche Datenbankrecherchen oder
�hnliches angebunden werden m�ssen. Ebenso besteht die M�glichkeit,
dass die Dienste des eigentlichen Bibliothekssystems nicht vollst�ndig
integriert sind und auch hier eine gegenseitige Anbindung erst
geschaffen werden muss.

Als Auswege werden u.a. monolithische Portall�sungen der jeweiligen
Anbieter von Bibliotheks\-sys\-temen gesehen, da diese naturgem�� die
Anbindung des wichtigen Dienstes Ausleihe mitbringen. Die Anbindung
dieses Dienstes kann aufgrund seiner generellen Wichtigkeit dann als
wesentlicher angesehen werden, als die Qualit�t und die F�higkeiten
des jeweiligen Portals.

Sinnvoller und mit deutlich mehr Freiheit bei der Auswahl des 'besten
Systems f�r den jeweiligen Zweck' verbunden ist die Kopplung
unterschiedlicher Systeme �ber geeignete Kommunikations\-schnitt\-stellen
herzustellen.

Ein international standardisiertes Verfahren, diese
Kommunikations\-schnitt\-stelle zu verwirklichen sind WebServices auf der
Basis von XML-RPC oder SOAP. Damit k�nnen verschiedene
Program\-mier\-spra\-chen verwendet werden.

Eine wesentliche Voraussetzung f�r die Integrationsf�higkeit ist die
Offenheit der Schnittstelle. Nur damit ist die Anbindung beliebiger
Systeme gew�hrleistet, w�hrend man bei der Verwendung propriet�rer
Protokolle den Interessen des jeweiligen Unternehmens ausgeliefert
ist.

Ein weiterer Vorteil einer offenen Schnittstelle ist die langfristige
Nutzung von Anwendungen, die auf Grundlage dieser Schnittstelle
programmiert wurden, auch bei m�glichen Systemwechseln. Lediglich die
Realisierung der Webservices f�r das jeweilige neue Bibliotheks\-system
ist dann zu �ndern.

Aus diesem 'Freiheitsgedanken' heraus und der entsprechenden konkreten
Problemstellung im OpenBib Recherche-Portal, bei der genau diese
Fragen bei der Anbindung des lokalen Bibliotheks\-systems eine Rolle
spielten, wurde mit der Implementierung einer freien, auf SOAP
basierenden WebServices-Schnittstelle begonnen.

Die Open Library WebServices sind geboren.

Wie schon OpenBib wurden Sie urspr�nglich als Freizeitprojekt unter
der GPL als OpenSource-Software begonnen und werden derzeit im
Wesentlichen auch in diesem Rahmen weiterentwickelt. Geplant ist
jedoch die Ausweitung auf weitere interessierte Entwickler, um eine
allgemeine Nut\-zung der Schnittstelle insbesondere im Hinblick auf
zus�tzliche Funktionalit�ten weiter voran\-zu\-treiben.

Insofern ist jeder herzlich eingeladen, bei der Weiterentwicklung und
Definition der Schnittstelle mitzuwirken.

\chapter{Referenz}


\section{Implementierung}

Die Open Library WebServices, kurz OLWS, wurden mit dem Perl-Modul
\texttt{SOAP::Lite} entwickelt. Damit ist eine Implementierung
weiterer Funktionen sehr einfach m"oglich. Durch die Verwendung von
SOAP kann jedoch prinzipiell auch jede andere Programmiersprache
genutzt werden, um die WebServices anzusprechen bzw. die bestehenden
WebServices zu reimplementieren.

\section{WebServices}

\subsection{urn:/Authentication}

\subsubsection*{Beschreibung}
"Uber dieses WebService wird die Authentifizierung am jeweiligen
System gesteuert.

\begin{tabular}{lp{10cm}}
  Unterst"utzte Zielsysteme & \texttt{Sisis}\\
  &\\
  Implementiert "uber die Module & \texttt{OLWS::Sisis::Authentication.pm}\\
\end{tabular}

\subsubsection{Funktionen}

\begin{tabular}{lp{10cm}}
  Funktionsname & \texttt{authenticate\_user}\\
  \hline
  Argumente     & \texttt{(Benutzername, Passwort, Datenbankname})\\
  R"uckgabe bei Erfolg & Benutzerinformationen:\\
  & Vorname, Nachname, Geschlecht, Geburtsdatum, Ort, Strasse, PLZ,
  Soll, Guthaben, Avanz, Branz, Bestellanz, Pflanz, Vmanz, Maanz,
  Vlanz, Sperre, Sperrdatum, Email\\
  
  
  &\\
  % Fehlercodes: &\\
  
  % \texttt{100-199} & Authentifizierungsfehler\\
  % \texttt{200-299} & Datenbankfehler\\
\end{tabular}

\subsection{urn:/Circulation}

\subsubsection*{Beschreibung}

"Uber dieses WebService werden benutzerkonto- bzw. ausleihspezifische
Funktionen gesteuert.

\begin{tabular}{lp{10cm}}
  Unterst"utzte Zielsysteme & \texttt{Sisis}\\
  &\\
  Implementiert "uber die Module & \texttt{OLWS::Sisis::Circulation.pm}\\
\end{tabular}

\subsubsection{Funktionen}

\begin{tabular}{lp{10cm}}
  Name & \texttt{get\_borrows}\\
  Beschreibung & Lieferung von Ausleihinformationen zu
  einem authentifizierten Nutzer einer spezifischen Zieldatenbank\\
  \hline
  Argumente     & \texttt{(Benutzername, Passwort, Datenbankname)}\\
  R"uckgabe bei Erfolg & Array mit Ausleihinformationen:\\
  & Katkey, Signatur, AusleihDatum, RueckgabeDatum\\
  &\\
  % Fehlercodes: &\\
  
  % \texttt{100-199} & Authentifizierungsfehler\\
  % \texttt{200-299} & Datenbankfehler\\
\end{tabular}

\begin{tabular}{lp{10cm}}
  Name & \texttt{get\_reservations}\\
  Beschreibung & Lieferung von Vormerkungsinformationen zu
  einem authentifizierten Nutzer einer spezifischen Zieldatenbank\\
  \hline
  Argumente     & \texttt{(Benutzername, Passwort, Datenbankname)}\\
  R"uckgabe bei Erfolg & Array mit Vormerkungsinformatione:\\
  & Katkey, Mediennummer, VormerkDatum, AufrechtDatum, Stelle\\
  &\\
  % Fehlercodes: &\\
  
  % \texttt{100-199} & Authentifizierungsfehler\\
  % \texttt{200-299} & Datenbankfehler\\
\end{tabular}

\begin{tabular}{lp{10cm}}
  Name & \texttt{get\_reminders}\\
  Beschreibung & Lieferung von "Uberziehungsinformationen zu
  einem authentifizierten Nutzer einer spezifischen Zieldatenbank\\
  \hline
  Argumente     & \texttt{(Benutzername, Passwort, Datenbankname)}\\
  R"uckgabe bei Erfolg & Array mit "Uberziehungsinformationen:\\
  & Katkey, Mediennummer, AusleihDatum, Leihfristende, Mahngebuehr, Saeumnisgebuehr\\
  &\\
  % Fehlercodes: &\\
  
  % \texttt{100-199} & Authentifizierungsfehler\\
  % \texttt{200-299} & Datenbankfehler\\
\end{tabular}

\subsection{urn:/Media}

\subsubsection*{Beschreibung}

"Uber dieses WebService wird der Zugriff auf die Katalogdaten im
Zielsystem gesteuert.

\begin{tabular}{lp{10cm}}
  Unterst"utzte Zielsysteme & \texttt{Sisis}\\
  &\\
  Implementiert "uber die Module & \texttt{OLWS::Sisis::Media.pm}\\
\end{tabular}

\subsubsection{Funktionen}

\begin{tabular}{lp{10cm}}
  Name & \texttt{get\_native\_title\_by\_katkey}\\
  Beschreibung & Lieferung der Titelinformationen zu einem Katkey
  direkt aus den nativen Datenbank-Blobs\\
  \hline
  Argumente     & \texttt{(Datenbankname, Katkey)}\\
  R"uckgabe bei Erfolg & Titeldaten im jeweiligen nativen Kategorienschema\\
  &\\
  % Fehlercodes: &\\
  
  % \texttt{100-199} & Authentifizierungsfehler\\
  % \texttt{200-299} & Datenbankfehler\\
\end{tabular}

\subsection{urn:/MediaStatus}

\subsubsection*{Beschreibung}

"Uber dieses WebService wird der Medienausleihstatus zu den jeweiligen
Medien geliefert.

\begin{tabular}{lp{10cm}}
  Unterst"utzte Zielsysteme & \texttt{Sisis}\\
  &\\
  Implementiert "uber die Module & \texttt{OLWS::Sisis::MediaStatus.pm}\\
\end{tabular}

\subsubsection{Funktionen}

\begin{tabular}{lp{10cm}}
  Funktionsname & \texttt{get\_mediastatus}\\
  \hline
  Argumente     & \texttt{(Katkey, Datenbankname)}\\
  R"uckgabe bei Erfolg & Ausleihstatus eines Mediums\\
  & Signatur, Exemplar, Standort, Status, Rueckgabe\\
  &\\
  % Fehlercodes: &\\
  
  % \texttt{100-199} & Authentifizierungsfehler\\
  % \texttt{200-299} & Datenbankfehler\\
\end{tabular}
 

\chapter{Programmierbeispiele}

An dieser Stelle werden zuk"unftig Programmierbeispiele in Perl genannt.

\section{Client}

An dieser Stelle werden Programmierbeispiele f"ur die Implementation
eines Clients gegeben, der auf einen spezifischen WebService zugreift.

\section{Server}

An dieser Stelle werden Programmierbeispiele f"ur die Implementation
eines Servers gegeben, der spezifische WebServices
bereitstellt. Generell kann der Server als CGI-Programm, als
eigenst"andiger D"amon oder unter mod\_perl laufen. Aus
Performance-Gr"unden sollte auf die CGI-Variante im Produktionsbetrieb
verzichtet werden. F"ur den Testbetrieb ist sie jedoch sehr gut geeignet.

\end{document}